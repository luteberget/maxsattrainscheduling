\documentclass[conference]{IEEEtran}
\usepackage[backend=bibtex,bibencoding=ascii,style=ieee,sorting=none]{biblatex}
\addbibresource{references.bib}


\begin{document}

\title{Incremental MaxSAT benchmarks from dynamic discretizations of train scheduling problems}

\author{\IEEEauthorblockN{Bjørnar Luteberget}
\IEEEauthorblockA{\textit{SINTEF Digital AS} \\
Oslo, Norway \\
bjornar.luteberget@sintef.no}
}
\maketitle

SAT and MaxSAT solvers are good choices for solving scheduling problems when there is a
natural and coarse discretization of time, such as in academic timetabling \cite{asin2014curriculum}.
%
In the railway domain, online/real-time scheduling applications have typically not been
a good match for time discretization approaches (see \cite{mannino2009optimal, harrod2011modeling, boland2019perspectives}), because there is
often no reasonable choice of coarseness: when the discretization is too fine,
the problem instance becomes too large to solve in a reasonable amount of time, and when the discretization is
too coarse, the approximate solutions may prove suboptimal or even infeasible
in the field.

Dynamic Discretization Discovery (DDD), recently introduced by \cite{boland2017continuous}, is a
technique used to convert scheduling problems over continuous time into
discrete variables in a way that mitigates these size and approximation issues. 
By solving a sequence of smaller problems, the discretization
is built incrementally in a way that ensures convergence to an optimal
solution of the complete continuous formulation.

In \cite{croella2024maxsat}, the DDD technique was adapted to a the train dispatching problem, which
is a type of railway scheduling problem that is solved in an online
setting where some trains have become delayed and are no longer able to follow
their original timetables.  The objective of the train dispatching problem in
\cite{croella2024maxsat} is to reschedule the trains to minimize the sum of delays.
%
The measure of the delay may either be the continuous numerical value of the
delay, or a step-wise function that defines constant  costs for exceeding some
defined delay thresholds. For example, if a train is delayed by 3 minutes it could incur
a cost of 1, if it is more than 10 minutes delayed, a cost of 2, and so on.

The paper \cite{croella2024maxsat} compares two approaches
for solving the train dispatching problem:
the commercial mixed-integer programming solver 
Gurobi using continuous variables, and
the 
dynamic discretization (DDD) solved using
 the RC2  MaxSAT algorithm \cite{IgnatievRC2}.
On step-wise objective functions, the MaxSAT approach had better performance.

The IPAMIR Train Scheduling benchmark submitted to the MaxSAT Evaulation 2024
is a command-line application that solves the train scheduling problems from
\cite{croella2024maxsat} with a step-wise objective function.
The program uses incremental calls to a MaxSAT solver through the IPAMIR
interface.  It calls IPAMIR to add hard clauses and to set literal weights (but
only on literals that previously did not have an assigned weight, i.e., the
weights are
non-decreasing).
The benchmarking program uses the \emph{linear rounded} objective defined in 
\cite{croella2024maxsat}, i.e. the cost is $c(d) =  \lfloor d / Q \rfloor $,
for each train's delay $d$, with $Q=180$ sec.

The input files provided with the benchmarking program 
are 72 problem instances constructed from real-world railway data
from two single-track
railroad networks, named Line A and Line B, 
extracted from the real-time train information system of the Norwegian state railways. 
Line A is 124 km long, includes 30 stations and
an average of 20 trains per problem instance.
Line B is 115 km, includes 20 stations and an average of 11 trains per problem instance.
There are 24 original instances from the information system, 
named in the style \texttt{origA01.txt}, where \texttt{A} is the 
railway line and \texttt{01} is the instance number.
In addition, two types of modifications were made to create harder problem instances:
instances named \texttt{track*.txt} have increased traveling time between stations,
and instances named \texttt{station*.txt} have increased station dwelling times.
See \cite{croella2024maxsat} for more details.
 


\printbibliography



\end{document}
